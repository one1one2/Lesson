\documentclass[a4paper,  11pt]{ctexart}
\usepackage{srcltx,graphicx}
\usepackage{amsmath, amssymb, amsthm}
\usepackage{color}
\usepackage{algorithm}
\usepackage{algorithmic}
\usepackage{setspace}
\usepackage{lscape}
\usepackage{multirow}
\usepackage{psfrag}
\usepackage{diagbox}
\usepackage{multirow}
\usepackage[hang]{subfigure}
\usepackage{float}
\usepackage[colorlinks,linkcolor=red,anchorcolor=blue,citecolor=green]{hyperref}

\newtheorem{theorem}{Theorem}
\newtheorem{lemma}{Lemma}
\newtheorem{definition}{Definition}
\newtheorem{comment}{Comment}
\newtheorem{conjecture}{Conjecture}

\newcommand\bbR{\mathbb{R}}
\newcommand\bbN{\mathbb{N}}
\newcommand\bbC{\mathbb{C}}
\newcommand\bx{\boldsymbol{x}}
\newcommand\dd{\,\mathrm{d}}

\newcommand\diag{\mathrm{diag}}
\newcommand\tr{\mthrm{tr}}

\setlength{\oddsidemargin}{0cm}
\setlength{\evensidemargin}{0cm}
\setlength{\textwidth}{150mm}
\setlength{\textheight}{230mm}

\newcommand\note[2]{{{\bf #1}\color{red} [ {\it #2} ]}}
%\newcommand\note[2]{{ #1 }} % using this line in the formal version

\newcommand\pd[2]{\dfrac{\partial {#1}}{\partial {#2}}}
\newcommand\od[2]{\dfrac{\dd {#1}}{\dd {#2}}}
\newcommand{\bm}[1]{\mbox{\boldmath{$#1$}}}
\renewcommand{\algorithmicrequire}{\textbf{Input:}} 
\renewcommand{\algorithmicensure}{\textbf{Output:}}

\begin{document}
\title{最优化理论与方法第三次上机作业}
\author{郑灵超\thanks{Email: lczheng@pku.edu.cn}\quad 1601110040}
\maketitle
%\tableofcontents
%\newpage
\section{上机作业介绍}
Minimax 问题是最优化一类经典的最优化问题,其描述如下:
\begin{equation}
  \min_{x} \max_{1\leq i \leq m} f_i(x).
\end{equation}
由于对象函数
\begin{equation}
  F(x) = \max_{1\leq i \leq m}f_i(x) 
\end{equation}
并不一定是可微函数,传统的优化方法往往难以奏效,我们需要为其单独设计算
法进行优化。
\section{算法介绍}
\subsection{光滑函数方法}
在\cite{smooth}中,作者提出引入一个光滑函数来逼近目标函数$F(x)$,
\begin{equation}
  f(x,\mu) = \mu \ln \sum_{i=1}^m \exp\left(\frac{f_i(x)}{\mu}\right)
\end{equation}
不难证明$f(x,\mu)$关于$\mu$单调递增,且
\[ 
   F(x)\leq f(x,\mu) \leq F(x) + \mu \ln m,
\]
因此当$\mu\rightarrow 0$时,可以用$f(x,\mu)$给出目标函数一个较好的逼近。
因此我们对$\mu\rightarrow 0$时的$f(x,\mu)$进行优化操作,就能实现对原目
标函数的优化。
\subsubsection*{算法说明}
\begin{enumerate}
  \item 设置初始点$x_0$,初始参数$\mu>0,0<\beta<1$.
  \item 利用Newton方法计算此时的负梯度方向:
    \begin{equation}
      \nabla_x^2f(x,\mu)d = -\nabla_x f(x,\mu).
    \end{equation}
  \item 利用线搜索计算步长$\alpha$.
  \item $x=x+\alpha d$.
  \item $\mu = \mu\beta$,若满足终止条件,则终止;否则回到第二步。
\end{enumerate}
\subsection{Least-p方法}
在\cite{leastp}中,作者对
Least-p方法进行了改进,采用了另外一种函数逼近形式

\begin{equation}
  U(x,u,p,\xi) = \left\{ 
  \begin{aligned}
  &M(x,\xi)\left(\sum_{i\in S(x,\xi)} 
  u_i\left(\frac{f_i(x)-\xi)}{M(x,\xi)}\right)^q
  \right)^{1/q},\quad &M(x,\xi)\neq 0\\ 
  &0 \quad & M(x,\xi)=0.
\end{aligned}
  \right.
\end{equation}
其中 
\begin{align}
  u_i\geq 0,\quad \sum_{i\in I} u_i = 1,\quad q =
  p\times\text{sgn}(M(x,\xi)),
\end{align}
而
\begin{align}
   M(x,\xi)=F(x)-\xi,\quad S(x,\xi) = \{i\in I| (f_i(x)-\xi)M(x,\xi) >
   0\}.
\end{align}
\subsubsection*{算法说明}
\begin{enumerate}
  \item 设置初始$p\geq 1,c\geq 1,u,\xi,x_0$
  \item 在当前参数下极小化逼近函数。
  \item 重新设置参数
    \[
       u_i = \frac{v_i}{\sum_{i\in I} v_i}, \quad i\in I,
    \]
    其中
    \[
         v_i = \left\{
         \begin{aligned}
           &u_i\left( \frac{f_i(x)-\xi}{M(x,\xi)}\right)^{q-1},\quad
           &i\in S(x,\xi),\\ 
           &0,\quad &i\in I - S(x,\xi).
         \end{aligned}
         \right.
    \]
  \item 若满足终止条件,则终止;否则$p=cp$,回到第二步.
\end{enumerate}
\subsection{SQP方法}
我采用了\cite{SQP}中提出的用于求解Minimax问题的SQP算法,其算法如下:
\subsubsection*{算法介绍}
\begin{enumerate}
  \item 选取初始点$X_0$,初始正定对称矩阵$B=I$,取
    $0<\sigma<\frac{1}{2},0<\delta\leq 1,\varepsilon\geq 0$.
  \item 求解二次规划问题
    \begin{align}
      \begin{aligned}
        &\min \quad  \frac 12 d^TBd + \frac{\delta}{2}t^2 + t,\\
        &s.t. \quad \nabla f_i(x)^Td-t\leq F(x)-f_i(x),\quad 1\leq
        i\leq m.
      \end{aligned}
    \end{align}
  其最优解设为$(d,t)$,相应的Lagrange乘子为$\lambda$,置
  \[ 
    d = \frac{d}{1+\delta t},\quad \lambda = \frac{\lambda}{1+\delta
    t}.
  \]
\item 若$\Vert d\Vert\leq \varepsilon$,则终止,否则取$0<\beta<1$
\item 计算最小的满足
  \begin{align}
    F(x+\beta^jd)\leq F(x) + \sigma\beta^j t
  \end{align}
  的自然数$j$,取$x' = x+\beta^jd$.
  \item 修正$B$,
    \begin{align*}
      &s = x'-x,\\
      &y = \sum_{i=1}^m \lambda_i \left[
      \nabla f_i(x')-\nabla f_i(x)
      \right],\\
      &y = \theta y+(1-\theta)Bs,
    \end{align*}
    其中 
    \begin{align*}
      \theta = \left\{
      \begin{aligned}
        &1,\quad &y's\geq 0.2s'Bs,\\
        &\frac{0.8s'Bs}{s'Bs-y's},\quad &otherwise.
      \end{aligned}
      \right.
    \end{align*}
    令
    \begin{align}
      B = B - \frac{Bss'B}{s'Bs}+\frac{yy'}{y's}.
    \end{align}
    转步2.
\end{enumerate}
\section{数值结果}
本次上机报告我们测试了三个案例:
\subsection{例1}
第一个例子是\cite{smooth}中的exmaple 1,三种方法的计算结果如下:
\begin{itemize}
  \item 光滑函数方法: 
    在21步迭代后到达解$(1.1390,0.8996)^T$,函数调用次数为84.
  \item Least-p方法:
    在22步迭代后到达解$(1.1390,0.8996)^T$,函数调用次数为88.
  \item SQP方法:
    在8步迭代后到达解$(1.1390,0.8996)^T$,函数调用次数为12.
\end{itemize}
\subsection{例3}
第一个例子是\cite{smooth}中的exmaple 3,即Rosen-Suzuki 问题
,三种方法的计算结果如下:
\begin{itemize}
  \item 光滑函数方法: 
    在24步迭代后到达解$(0,1,2,-1)^T$,函数调用次数为110.
  \item Least-p方法:
    在17步迭代后到达解$(0,1,2,-1)^T$,函数调用次数为79.
  \item SQP方法:
    在22步迭代后到达解$(0,1,2,-1)^T$,函数调用次数为150.
\end{itemize}
\subsection{滤波器问题}
第一个例子是\cite{leastp}中的digital filter问题,三种方法的计算结果如下:
\begin{itemize}
  \item 光滑函数方法: 
  无法计算。
  \item Least-p方法:
  无法计算。
  \item SQP方法:
    在39步迭代后到达解
    $(0.0001,0.9801,0.0000,-0.1657,0.0001,-0.7351,\\
    0.0001,-0.7672,0.3679)^T$,
    函数调用次数为357,函数值$f=0.006189$.
\end{itemize}



\section{上机报告总结}
本次上机报告,我们实现了三种梯度型下降方法,在此作一个总结:
\begin{enumerate}
\item 光滑函数法和Least-p法都是利用一个带参数的光滑函数来逼近目标函数
  ,再令参数趋于0或无穷,来实现对目标函数的逼近。可能是由于我们在求解其子问
  题时采用Newton迭代法,对较为复杂的滤波器问题,这两个方法的表现并不好
  。
\item SQP方法是用于求解二次规划问题的经典方法,我们这里采用了matlab自
  带的quadprog函数求解它的二次规划子问题。该方法对测试的三个例子均有不
  错的表现。
\end{enumerate}
\renewcommand\refname{参考文献}
\begin{thebibliography}{99}
    \bibitem{smooth}Song Xu. Smoothing Method for Minimax Problems.
    Computational Optimization and Applications, 20, 267–279, 2001.
  \bibitem{leastp}
    Christakis CHARALAMBOUS. 
    ACCELERATION OF THE LEAST pth ALGORITHM
FOR MINIMAX OPTIMIZATION WITH ENGINEERING
APPLICATIONS.Mathematical Programming 17 (1979) 270-297. 
\bibitem{SQP}薛毅.求解Minimax优化问题的SQP方法.系统科学与数学, 22, 
  355-364, 2002.
\end{thebibliography}

\end{document}

