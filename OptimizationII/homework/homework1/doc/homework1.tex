\documentclass[a4paper,  11pt]{ctexart}
\usepackage{srcltx,graphicx}
\usepackage{amsmath, amssymb, amsthm}
\usepackage{color}
\usepackage{algorithm}
\usepackage{algorithmic}
\usepackage{setspace}
\usepackage{lscape}
\usepackage{multirow}
\usepackage{psfrag}
\usepackage{diagbox}
\usepackage{multirow}
\usepackage[hang]{subfigure}
\usepackage{float}
\usepackage[colorlinks,linkcolor=red,anchorcolor=blue,citecolor=green]{hyperref}

\newtheorem{theorem}{Theorem}
\newtheorem{lemma}{Lemma}
\newtheorem{definition}{Definition}
\newtheorem{comment}{Comment}
\newtheorem{conjecture}{Conjecture}

\newcommand\bbR{\mathbb{R}}
\newcommand\bbN{\mathbb{N}}
\newcommand\bbC{\mathbb{C}}
\newcommand\bx{\boldsymbol{x}}
\newcommand\dd{\,\mathrm{d}}

\newcommand\diag{\mathrm{diag}}
\newcommand\tr{\mthrm{tr}}

\setlength{\oddsidemargin}{0cm}
\setlength{\evensidemargin}{0cm}
\setlength{\textwidth}{150mm}
\setlength{\textheight}{230mm}

\newcommand\note[2]{{{\bf #1}\color{red} [ {\it #2} ]}}
%\newcommand\note[2]{{ #1 }} % using this line in the formal version

\newcommand\pd[2]{\dfrac{\partial {#1}}{\partial {#2}}}
\newcommand\od[2]{\dfrac{\dd {#1}}{\dd {#2}}}
\newcommand{\bm}[1]{\mbox{\boldmath{$#1$}}}
\renewcommand{\algorithmicrequire}{\textbf{Input:}} 
\renewcommand{\algorithmicensure}{\textbf{Output:}}

\begin{document}
\title{最优化理论与方法第一次上机作业}
\author{郑灵超\thanks{Email: lczheng@pku.edu.cn}\quad 1601110040}
\maketitle
%\tableofcontents
%\newpage
\section{上机作业介绍}
本次上机作业的任务是编写带步长Newton方法,稳定Newton方法和
More-Sorensen方法的代码,并在几个测试函数上检验他们的性质。
\section{算法介绍}
\subsection{带步长Newton方法}
带步长的Newton方法是在传统的Newton方法的基础上,不再限制迭代步长为1,
而利用线搜索来寻找合适的步长,其算法为:
\begin{algorithm}[H]
\setstretch{1.35} 
\caption{带步长Newton方法}
\begin{algorithmic}[2]
\REQUIRE ~~\\
函数信息,包括函数值$f$,梯度值$g$和Hesse矩阵$G$; \quad 初始点$x_0$;

\ENSURE ~~\\
最小值点$x$,函数值$f(x)$,迭代次数iter,函数调用次数feva;

\hspace{-0.65cm}\textbf{Procedure:}

\textbf{Step 1:}  初始设置$k = 0$, $\varepsilon_f>0$; \\
\textbf{Step 2:}  计算$x_k$处的函数值,梯度和Hesse矩阵$f_k,g_k,G_k$; \\
\textbf{Step 3:}  若$k>0$且$|f_{k}-f_{k-1}|<\varepsilon_f$,循环终止,并输出
$x_k,f_k$;\\
\textbf{Step 4:}  计算Newton方法的下降方向$d_k$: $G_kd_k=-g_k$;\\
\textbf{Step 5:}  利用线搜索计算步长$\alpha_k$;\\
\textbf{Step 6:}  计算$x_{k+1}=x_k+\alpha_kd_k$; \\
\textbf{Step 7:}  令$k=k+1$,回到\textbf{Step 2}.
\end{algorithmic}
\end{algorithm}
带步长Newton方法在传统的Newton方法的基础上引入了线搜索的步长,得到了比
传统Newton方法更广泛的应用。但当Hesse矩阵不正定时,可能会出现下降方向
无法计算的情况。
\subsection{稳定Newton方法}
稳定Newton方法利用了修正Cholesky分解
\footnote{修正Cholesky分解参见\cite{article1}的Algorithm 3.3.2.}
,使得计算所得的下降方向始终满足
$g^Td<0$,其算法\footnote{\cite{article1}中Algorithm 3.5.4}为:
\begin{algorithm}[H]
\setstretch{1.35} 
\caption{稳定Newton方法}
\begin{algorithmic}[2]
\REQUIRE ~~\\
函数信息,包括函数值$f$,梯度值$g$和Hesse矩阵$G$; \quad 初始点$x_0$;

\ENSURE ~~\\
最小值点$x$,函数值$f(x)$,迭代次数iter,函数调用次数feva;

\hspace{-0.65cm}\textbf{Procedure:}

\textbf{Step 1:}  初始设置$k = 0$, $\varepsilon_f,\varepsilon_g>0$; \\
\textbf{Step 2:}  计算$x_k$处的函数值,梯度和Hesse矩阵$f_k,g_k,G_k$; \\
\textbf{Step 3:}  若$k>0$且$|f_{k}-f_{k-1}|<\varepsilon_f$,循环终止,并输出
$x_k,f_k$;\\
\textbf{Step 4:}  计算Hesse矩阵$G_k$的修正Cholesky分解
:
$G_k+E_k=L_kD_kL^T_k$;\\
\textbf{Step 5:}  计算下降方向$d_k$:\\
\hspace{1.52cm}若$\Vert g_k\Vert>\varepsilon_g,
L_kD_kL^T_kd_k=-g_k$;\\
\hspace{1.52cm}
否则设$D_{j,j}-E_{j,j}$当$j=t$时取到最小值$\varphi$,\\
\hspace{1.52cm}
若$\varphi\geq0$,终止并输出;否则求解$L^T_kd_k=e_t$;\\

\textbf{Step 6:}  对$d_k$进行修正:若$g_k^Td_k>0$,取$d_k=-d_k$;\\
\textbf{Step 7:}  利用线搜索计算步长$\alpha_k$;\\
\textbf{Step 8:}  计算$x_{k+1}=x_k+\alpha_kd_k$; \\
\textbf{Step 9:}  令$k=k+1$,回到\textbf{Step 2}.
\end{algorithmic}
\end{algorithm}
稳定Newton方法利用修正Cholesky分解解决了当Hesse矩阵有零特征值和负特征
值而导致的下降方向无法计算的情况,得到了一个始终满足$g^Td\leq0$的下降方向
$d$,并在不定点处给出了二阶下降方向$d$,满足$d^TGd<0$.
\subsection{More-Sorensen方法}
More-Sorensen方法提出了二阶Wolfe准则:
\begin{align*}
  x(\alpha)=x_k+\alpha^2s_k+\alpha d_k,\quad (s_k,d_k)\text{是
  }x_k\text{处的下降对},
\end{align*}
满足
\begin{gather}
  \label{eq:wolfe1}
  f(x(\alpha))\leq f(x)+\rho\alpha^2\left[ \nabla f(x)^{T}s+\frac{1}{2}
  d^T\nabla^{2}f(x)d\right],\\
  \label{eq:wolfe2}
  \nabla f(x(\alpha))^Tx'(\alpha)\geq \sigma\left[ \nabla f(x)^Td+
  2\alpha\nabla f(x)^Ts+\alpha d^T\nabla^2f(x)d\right],
\end{gather}
其中$0<\rho\leq\sigma<1$.
\begin{algorithm}[H]
\setstretch{1.35} 
\caption{More-Sorensen方法}
\begin{algorithmic}[2]
\REQUIRE ~~\\
函数信息,包括函数值$f$,梯度值$g$和Hesse矩阵$G$; \quad 初始点$x_0$;

\ENSURE ~~\\
最小值点$x$,函数值$f(x)$,迭代次数iter,函数调用次数feva;

\hspace{-0.65cm}\textbf{Procedure:}

\textbf{Step 1:}  初始设置$k = 0$, $\varepsilon_f>0$; \\
\textbf{Step 2:}  计算$x_k$处的函数值,梯度和Hesse矩阵$f_k,g_k,G_k$; \\
\textbf{Step 3:}  若$k>0$且$|f_{k}-f_{k-1}|<\varepsilon_f$,循环终止,并输出
$x_k,f_k$;\\
\textbf{Step 4:}  计算下降对$(s_k,d_k)$;若下降对无法计算,终止并输出;\\
\textbf{Step 5:}  对$d_k$进行修正:若$g_k^Td_k>0$,取$d_k=-d_k$;\\
\textbf{Step 6:}  计算步长$\alpha_k$;\\
\textbf{Step 7:}  计算$x_{k+1}=x_k+\alpha_kd_k$; \\
\textbf{Step 8:}  令$k=k+1$,回到\textbf{Step 2}.
\end{algorithmic}
\end{algorithm}
其中下降对的计算方法参见\cite{article1}的162-163页,而步长我们选取
$\alpha=\gamma^i$中满足二阶Wolfe条件\eqref{eq:wolfe1}和
\eqref{eq:wolfe2}的$i$最小的一个,其中
$0<\gamma<1$为参数。

\section{程序说明}
本次提交的代码包含以下文件:
\begin{itemize}
  \item test.m: ~~ 运行文件,可以设置求解方法和求解问题。
  \item DampedNewton.m: ~~ 带步长Newton方法的函数。
  \item StableNewton.m: ~~ 稳定Newton方法的函数。
  \item ModCholesky.m ~~ 稳定Newton方法需要用到的修正Cholesky函数。
  \item Sorensen.m: ~~ More Sorensen方法的函数。
  \item DescentPair.m ~~ More Sorensen方法需要用到的计算下降对的函数。
\end{itemize}
设置的参数为
\[  
\varepsilon_f=10^{-25},\quad \varepsilon_g=10^{-15}.
\]
\section{数值结果}
\subsection{Watson function}
\begin{table}[H]
  \centering
  \begin{tabular}{|c|c|c|c|c|}
    \hline
  方法 &  n & f(x*) & 迭代次数 & 函数调用次数  \\
  \hline 
  \multirow{5}{*}{带步长Newton方法} & 6 & 2.3e-3 & 85 & 388 \\
  \cline{2-5}
  & 9& 1.3998e-6 & 109 & 500 \\
  \cline{2-5}
  & 12 & 4.7224e-10 & 103 & 286 \\
  \cline{2-5}
  & 15 & / & / & / \\
  \cline{2-5}
  & 18 & / & / & / \\
  \hline
  \multirow{5}{*}{稳定Newton方法} & 6 & 2.3e-3 & 62 & 462 \\
  \cline{2-5}
  & 9& 1.3998e-6 & 9 & 250 \\
  \cline{2-5}
  & 12 & 4.7224e-10 & 68 & 351 \\
  \cline{2-5}
  & 15 & 8.448e-13 & 52 & 1325 \\
  \cline{2-5}
  & 18 & 5.9097e-9 & 36 & 900 \\
  \hline
  \multirow{5}{*}{More-Sorensen方法} & 6 & 2.3e-3 & 13 & 14 \\
  \cline{2-5}
  & 9& 1.3998e-6 & 14 & 15 \\
  \cline{2-5}
  & 12 & 4.7224e-10 & 14 & 15 \\
  \cline{2-5}
  & 15 & 2.7500e-8 & 12 & 2051 \\
  \cline{2-5}
  & 18 & 1.0197e-7 & 11 & 2050 \\
  \hline
  
  \end{tabular}
\end{table}
此处我们对于步长Newton和稳定Newton均采用强Wolfe准则的线搜索,其中稳定
Newton在n=9时该方法失效,采用精确线搜索。

比较三个方法可以发现稳定Newton方法和More-Sorensen方法比起步长
Newton方法更具有稳定性,能解决Hesse矩阵接近奇异和不定的情形。
\subsection{Extended Powell singular function}
\begin{table}[H]
  \centering
  \begin{tabular}{|c|c|c|c|c|}
    \hline
  方法 &  n & f(x*) & 迭代次数 & 函数调用次数  \\
  \hline 
  \multirow{5}{*}{带步长Newton方法} 
  & 20 & 9.6476e-17 & 18 & 76 \\
  \cline{2-5}
  & 40& 1.9187e-17 & 19 & 80 \\
  \cline{2-5}
  & 60 & 2.8781e-17 & 19 & 80 \\
  \cline{2-5}
  & 80 & 3.8375e-17 & 19 & 80 \\
  \cline{2-5}
  & 100 & 4.7968e-17 & 19 & 80 \\
  \hline
  \multirow{5}{*}{稳定Newton方法} 
  & 20 & 9.6476e-17 & 18 & 76 \\
  \cline{2-5}
  & 40& 1.9187e-17 & 19 & 80 \\
  \cline{2-5}
  & 60 & 2.8781e-17 & 19 & 80 \\
  \cline{2-5}
  & 80 & 3.8375e-17 & 19 & 80 \\
  \cline{2-5}
  & 100 & 4.7968e-17 & 19 & 80 \\
  \hline
  \multirow{5}{*}{More-Sorensen方法} 
  & 20 & 2.2507e-16 & 25 & 27 \\
  \cline{2-5}
  & 40& 9.2510e-17 & 27 & 28 \\
  \cline{2-5}
  & 60 & 1.3877e-16 & 27 & 28 \\
  \cline{2-5}
  & 80 & 1.8502e-16 & 27 & 28 \\
  \cline{2-5}
  & 100 & 2.3128e-16 & 27 & 28 \\
  \hline
  
  \end{tabular}
\end{table}
该问题三种方法都可以得到不错的结果,说明该问题本身的性质良好,而且
该问题我们对步长Newton方法和稳定Newton方法都采用相同的非精确线搜索,
可以发现这两种方法的计算结果完全一致,说明Hesse矩阵是一个正定矩阵。
\subsection{Biggs EXP6 function}
对于$m=13$的BiggsEXP6问题,我们的计算结果为:
\begin{itemize}
  \item 步长Newton方法:无法收敛到正确的解。
  \item 稳定Newton方法:利用精确线搜索可以收敛,迭代次数37,函数调用次
    数925,$x=(1,10,1,5,4,3)^T$,$f=2.3190e-18$.
  \item More-Sorensen方法:可以收敛到正确解$x=(4,10,3,5,1,1)^T$,迭代次
    数为25,函数调用次数39,$f=1.4785e-19$.
\end{itemize}
%\subsection{Discrete boundary value function}

\section{上机报告总结}
本次上机报告,我们实现了三种梯度型下降方法,在此作一个总结:
\begin{enumerate}
  \item 带步长Newton方法是普通Newton方法利用线搜索的一个推广,
    能解决Hesse矩阵正定情况的问题,但对较为奇异的Hesse矩阵和不定的
    Hesse矩阵比较乏力。
  \item 稳定Newton方法利用了修正Cholesky分解,能够解决Hesse矩阵不正定
    情况下的问题。
  \item More-Sorensen方法利用了下降对的性质,能够解决Hesse矩阵不
    定的问题,但对于接近0的特征值没有较好的机算结果。
  \item 改进的两种计算方法无论是在计算结果,还是在迭代效率上,都比起初
    始的Newton方法具有不小的提高。
\end{enumerate}
\renewcommand\refname{参考文献}
\begin{thebibliography}{99}
\bibitem{article1}Wenyu Sun and Ya-Xiang Yuan.
  Optimization theory and methods: nonlinear
  programming. Springer, 2006.
\end{thebibliography}

\end{document}

