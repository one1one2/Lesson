\documentclass[a4paper,  11pt]{ctexart}
\usepackage{srcltx,graphicx}
\usepackage{amsmath, amssymb, amsthm}
\usepackage{color}
\usepackage{lscape}
\usepackage{multirow}
\usepackage{psfrag}
\usepackage{diagbox}
\usepackage[hang]{subfigure}
\usepackage{float}
\usepackage[colorlinks,linkcolor=black,anchorcolor=blue,citecolor=green]{hyperref}

\newtheorem{theorem}{Theorem}
\newtheorem{lemma}{Lemma}
\newtheorem{definition}{Definition}
\newtheorem{comment}{Comment}
\newtheorem{conjecture}{Conjecture}

\newcommand\bbR{\mathbb{R}}
\newcommand\bbN{\mathbb{N}}
\newcommand\bbC{\mathbb{C}}
\newcommand\bx{\boldsymbol{x}}
\newcommand\dd{\,\mathrm{d}}

\newcommand\diag{\mathrm{diag}}
\newcommand\tr{\mthrm{tr}}

\setlength{\oddsidemargin}{0cm}
\setlength{\evensidemargin}{0cm}
\setlength{\textwidth}{150mm}
\setlength{\textheight}{230mm}

\newcommand\note[2]{{{\bf #1}\color{red} [ {\it #2} ]}}
%\newcommand\note[2]{{ #1 }} % using this line in the formal version

\newcommand\pd[2]{\dfrac{\partial {#1}}{\partial {#2}}}
\newcommand\od[2]{\dfrac{\dd {#1}}{\dd {#2}}}
\newcommand{\bm}[1]{\mbox{\boldmath{$#1$}}}

\begin{document}
\title{利用HYPRE并行求解三维热方程 \\ 并行计算第四次上机作业}
\author{郑灵超}
\maketitle
\tableofcontents
\newpage
\section{问题介绍}
本次要求解的方程为三维的热方程问题,其定义域为八面体
\[
 |x|+|y|+|z|\leq 1.
\]
方程形式和边界条件分别为
\begin{equation}
	\label{eq:Heatequation}
	\left\{
	\begin{aligned}
	&u_t-\Delta u = f,  \\
	&u|_{t=0} = u_0, \\
	&u|_{|x|+|y|+z=1} = g_{up}, \\
	&\pd{u}{\bm n}|_{|x|+|y|-z=1} = g_{down}. 
\end{aligned}
\right.
\end{equation}

\section{算法介绍}
此次我们采用的算法是隐式差分格式,将求解区域按正方形网格剖分,即网格尺
度$h=\frac 1 N$,则网格节点为
$x=ih,y=jh,z=kh$
其迭代格式为
\begin{equation}
	\begin{aligned}
		\label{eq:iteration}
		&\frac{u^{(t+\Delta t)}(ih,jh,kh) - u^{(t)}(ih,jh,kh)}{\Delta t} 
		+\frac{1}{h^2} [ u^{(t+\Delta t)}(ih+h,jh,kh) +u^{(t+\Delta
		t)}(ih-h,jh,kh) \\ 
 	&+u^{(t+\Delta t)}(ih,jh+h,kh) + u^{(t+\Delta t)}(ih,jh-h,kh)  
	+u^{(t+\Delta t)}(ih,jh,kh+h) \\ &+u^{(t+\Delta t)}(ih,jh,kh-h) - 6
	u^{(t+\Delta t)}(ih,jh,kh)] = f(ih,jh,kh,t+\Delta t)
\end{aligned}
\end{equation}
将其写成方程组的形式:
\begin{equation}
    {\bf A} \bm x = \bm b,
\end{equation}
其中矩阵${\bf A}$是一个主对角线为$1+6\frac{\Delta t}{h^2}$,同一行还有
6个$-\frac{\Delta t}{h^2}$的七对角矩阵,向量$\bm b$为$f\Delta
t+u^{(t)}$。

这部分数值格式的精度为$O(\Delta t+h^2)$,隐式格式对CFL条件数
$\frac{\Delta t}{h^2}$没有要求,
因此这部分数值格式的精度为$O(\Delta t+h^2)$。

在上边界,即狄利克雷边界上,我们直接进行赋值;在下边界,即诺依曼边界
上,我们根据$t+\Delta t$时刻该点的法向导数,采用内部点和当前点的差来逼
近这个法向导数,得到一个关于网格密度$h$为一阶精度的数值格式,因此我么
最终采用的数值格式的精度为$O(\Delta t+h)$。
\section{程序分析}
\subsection{程序说明}
我们定义了一个类Heat,用于求解这一类型的热方程,用户可以通过在main.cpp
文件中修改方程的初边值条件。网格密度$N$和计算终止时间
$t_{end}$通过命令行参数读入。具体程序的结构和声明可以参见doc目录下的
refman.pdf 和 html 目录下的index.html网页。以下我们简要说明一下并行实
现的算法流程:
\begin{enumerate}
	\item 读入信息,如网格密度,计算终止时间,所用进程数目,CFL条件数
		,和设置的初边值条件。
	\item 由0号进程进行一些预处理工作:将三维的点用一个长度为$M$的
		std::vector表示
		,并给出将点转化为vector中下标的函数。并利用这个函数计算出
		每个编号对应的点的
		坐标和邻居的编号,并将这些信息发送给所需要的进程。
	\item 每个进程各自的初始化工作,包括计算每个进程所包含的点的起始编
		号和终止编号。我们这里将所有网格均等分给每个进程,第i号进程的
		需要处理的网格编号为$\frac{iM}{size}$到$\frac{(i+1)M}{size}$。
		此外,每个进程需要计算自己进行迭代计算时需要相邻进程提供的数据
		。
    \item 每个进程计算各自的矩阵${\bf A}$,并保存为~\verb|HYPRE_IJMatrix|~格
        式。此外,我们可以在此时初始化求解器,从而节省整体运行时间。
	\item 每个进程分别计算自己所管辖区域的$t=0$的初值情况。
    \item 进行一步迭代计算,包括从前一步的计算结果中获取向量$\bm u^{(t)}$
        ,计算得到向量$\bm b$,并利用HYPRE的求解方法来求解向量$\bm x$
        。
	\item 迭代时间达到终止时间,将解返回给用户,并进行释放内存,清除数
        据等操作。
\end{enumerate}
\subsection{程序评价}
\noindent
这次我们采用的程序的优点有:
\begin{itemize}
	\item 将整个数据拉成一个一维连续的vector,并等分给各个进程,负载较
		为均衡。
	\item 用类进行封装,用户只需要通过主函数进行修改。
    \item 由于每次迭代的矩阵${\bf A}$并没有变化,我们只对它进行了一次
        计算,减少了不少计算量。同时,在迭代开始之前就对求解器进行了定
        义,也可以节省运行时间。
\end{itemize}
此外,我认为此次写的程序还有如下不足:
\begin{itemize}
	\item 一些基本信息的初始化工作仍然由0号进程完成,这儿我认为还有改
		进空间。
	\item 对于边界,我们尚未给出一个精细的处理方法。目前求解方法的精度
        为$O(\Delta t+h)$,并不能令人满意。
\end{itemize}
\section{数值结果}
\subsection{误差分析}
我们选取了一个有精确解的方程进行计算,其精确解为
\begin{equation}
	\label{eq:testfunction}
	u(x,y,z,t)=\sin ((x^2+y^2+z^2)t)
\end{equation}
采取之前所介绍的方法进行计算,并将解与真实解做了误差比较,
我们得到如下的结果:
\begin{table}[H]
	\centering
	\begin{tabular}{ccccc}
		\hline 
		网格密度$N$ & L2误差 & 阶数 & CPU时间(s) & CPU时间的阶数 \\ 
		\hline 
		10  & 3.25e-3 & / & 0.03 & /  \\
		\hline 
		20 & 1.41e-3 & 1.2 & 0.23 & 2.94 \\
		\hline 
		40 & 6.57e-4 & 1.1 & 5.44 & 4.56 \\
		\hline 
		80 & 4.00e-4 & 0.7 & 181.64 & 5.06\\ 
		\hline
	\end{tabular}
\end{table}
此处我们采用的$\Delta t = 2h^2$,计算终止时间为0.1,进程数为4.

受到诺依曼边界条件的影响,我们的数值精度只有1阶。
\iffalse
\subsection{并行效率分析}
我们的测试函数仍然为\eqref{eq:testfunction},对不同的进程数,得到的计算
结果如下:
\begin{table}[H]
	\centering
	\begin{tabular}{ccccccccc}
	\hline
       进程数  & 1 & 2 & 3 & 4 & 5 & 6 & 7 & 8 \\
	\hline
	  CPU时间(s)& 4.99 & 4.56 & 17.8507 & 13.5908 & 14.0187 &
	  11.8521 & 10.2867 & 9.5289\\
	\hline 
	 加速比 & / & 1.91 &  2.85 & 3.74 & 3.63 & 4.29 & 4.94 &  5.34 \\
	\hline 
	 效率 & / & 95.7\%  & 95.0\% &  93.5\% & 72.5\% &71.5\% & 70.6\% & 
	 66.7\% \\
	 \hline 
 \end{tabular}
\end{table}
此处我们采用的$\Delta t = 2h^2$,计算终止时间为0.1.
\par
通过这些数据,我们可以发现该程序的并行效率较高,但当核数多时效率会有所
下降。经过比较,该程序最合适的进程数应设置为4.
\fi
\section{上机报告总结}
此次上机作业我们学习了HYPRE软件,并利用它重新求解了第二次作业中的热方
程。

此外,我们熟悉HYPRE中行压缩的稀疏矩阵的存储和赋值方式,了解了一些常用
的解线性方程组的求解器,对HYPRE这类数学软件的安装和使用有了一定认识。


\end{document}

