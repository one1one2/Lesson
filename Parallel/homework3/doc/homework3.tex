\documentclass[a4paper,  11pt]{ctexart}
\usepackage{srcltx,graphicx}
\usepackage{amsmath, amssymb, amsthm}
\usepackage{color}
\usepackage{lscape}
\usepackage{multirow}
\usepackage{psfrag}
\usepackage{diagbox}
\usepackage[hang]{subfigure}
\usepackage{float}
\usepackage[colorlinks,linkcolor=black,anchorcolor=blue,citecolor=green]{hyperref}

\newtheorem{theorem}{Theorem}
\newtheorem{lemma}{Lemma}
\newtheorem{definition}{Definition}
\newtheorem{comment}{Comment}
\newtheorem{conjecture}{Conjecture}

\newcommand\bbR{\mathbb{R}}
\newcommand\bbN{\mathbb{N}}
\newcommand\bbC{\mathbb{C}}
\newcommand\bx{\boldsymbol{x}}
\newcommand\dd{\,\mathrm{d}}

\newcommand\diag{\mathrm{diag}}
\newcommand\tr{\mthrm{tr}}

\setlength{\oddsidemargin}{0cm}
\setlength{\evensidemargin}{0cm}
\setlength{\textwidth}{150mm}
\setlength{\textheight}{230mm}

\newcommand\note[2]{{{\bf #1}\color{red} [ {\it #2} ]}}
%\newcommand\note[2]{{ #1 }} % using this line in the formal version

\newcommand\pd[2]{\dfrac{\partial {#1}}{\partial {#2}}}
\newcommand\od[2]{\dfrac{\dd {#1}}{\dd {#2}}}
\newcommand{\bm}[1]{\mbox{\boldmath{$#1$}}}

\begin{document}
\title{并行求解三维偏微分方程 \\ 并行计算第三次上机作业}
\author{郑灵超}
\maketitle
\tableofcontents
\newpage
\section{问题介绍}
本次要求解的方程是一个三维的偏微分方程,其定义域为立方体
\[
  |x|,|y|,|z|\leq \pi
\]
方程形式为
\begin{equation}
	\label{eq:PDEequation}
	-\Delta u + u^3 = f ,
\end{equation}
所给的边界条件为周期边界。
\section{算法介绍}

此次我们采用的算法是显式差分格式,将求解区域按正方形网格剖分,即网格尺
度$h=\frac 1 N$,则网格节点为
$x=ih,y=jh,z=kh$
其迭代格式为
\begin{equation}
	\begin{aligned}
		\label{eq:iteration}
	u^{(t+\Delta t)}(ih,jh,kh) =& u^{(t)}(ih,jh,kh) +\Delta t
	f(ih,jh,kh,t) 
	 +\frac{\Delta t}{h^2}[
	u^{(t)}(ih+h,jh,kh)  \\  
	& +u^{(t)}(ih-h,jh,kh) 
 	+u^{(t)}(ih,jh+h,kh)+u^{(t)}(ih,jh-h,kh) \\ 
	& +u^{(t)}(ih,jh,kh+h)+u^{(t)}(ih,jh,kh-h) - 6 u^{(t)}(ih,jh,kh)
	]
\end{aligned}
\end{equation}
这部分数值格式的精度为$O(\tau+h^2)$,而CFL条件要求
$$\frac{\tau}{h^2}\leq \frac{1}{6},$$
因此这部分数值格式的精度为$O(h^2)$。

在上边界,即狄利克雷边界上,我们直接进行赋值;在下边界,即诺依曼边界
上,我们根据$t$时刻该点的法向导数,构造出一些虚设的外部的点的函数值,
再利用\eqref{eq:iteration}进行迭代计算。例如在$x+y-z=1$的边界上,我们
采取的格式是,
\begin{equation}
	\label{eq:neumann}
	\begin{aligned}
	u^{(t+\Delta t)}(ih,jh,kh) =& u^{(t)}(ih,jh,kh) +\Delta t
	f(ih,jh,kh,t) 
	 +\frac{\Delta t}{h^2}[
	  2u^{(t)}(ih-h,jh,kh) \\
 	&+ 2u^{(t)}(ih,jh-h,kh)+2u^{(t)}(ih,jh,kh+h)-6 u^{(t)}(ih,jh,kh)\\
	&+ 2\sqrt{3}g_{down}(ih,jh,kh,t)]
	\end{aligned}
\end{equation}
\section{程序分析}
\subsection{程序说明}
我们定义了一个类Heat,用于求解这一类型的热方程,用户可以通过在main.cpp
文件中修改方程的初边值条件和CFL条件数。网格密度$N$和计算终止时间
$t_{end}$通过命令行参数读入。具体程序的结构和声明可以参见doc目录下的
refman.pdf 和 html 目录下的index.html网页。以下我们简要说明一下并行实
现的算法流程:
\begin{enumerate}
	\item 读入信息,如网格密度,计算终止时间,所用进程数目,CFL条件数
		,和设置的初边值条件。
	\item 由0号进程进行一些预处理工作:将三维的点用一个长度为$M$的
		std::vector表示
		,并给出将点转化为vector中下标的函数。并利用这个函数计算出
		每个编号对应的点的
		坐标和邻居的编号,并将这些信息发送给所需要的进程。
	\item 每个进程各自的初始化工作,包括计算每个进程所包含的点的起始编
		号和终止编号。我们这里将所有网格均等分给每个进程,第i号进程的
		需要处理的网格编号为$\frac{iM}{size}$到$\frac{(i+1)M}{size}$。
		此外,每个进程需要计算自己进行迭代计算时需要相邻进程提供的数据
		。
	\item 每个进程分别计算自己所管辖区域的$t=0$的初值情况。
	\item 进行一步迭代计算。
	\item 由0号进程收集终止时刻的所有值,合并成一个整体的向量,并返回
		给用户。
\end{enumerate}
其中进行一步迭代计算的流程为,这里的一步迭代是每个进程分别操作的。
\begin{enumerate}
\item  向相邻进程收取所需要的前一时刻的数据,与自己前一时刻的数据组合
	起一个完整的数据集合。
\item 利用迭代格式\eqref{eq:iteration}和\eqref{eq:neumann}进行迭代计算。
\item 将自己所被需要的信息发送给相邻的进程。
\end{enumerate}
\subsection{程序评价}
\noindent
这次我们采用的程序的优点有:
\begin{itemize}
	\item 将整个数据拉成一个一维连续的vector,并等分给各个进程,负载较
		为均衡。
	\item 精确计算了计算每个点所需要的信息,并向其它进程索取,保证了信
		息传递没有浪费。
	\item 用类进行封装,用户只需要通过主函数进行修改。
\end{itemize}
此外,我认为此次写的程序还有如下不足:
\begin{itemize}
	\item 一些基本信息的初始化工作仍然由0号进程完成,这儿我认为还有改
		进空间。
	\item 对于边界,我们尚未给出一个精细的处理方法。
	\item 采用了显式求解格式,使得时间步长受到限制。
\end{itemize}
\section{数值结果}
\subsection{误差分析}
我们选取了一个有精确解的方程进行计算,其精确解为
\begin{equation}
	\label{eq:testfunction}
	u(x,y,z,t)=\sin ((x^2+y^2+z^2)t)
\end{equation}
采取之前所介绍的方法进行计算,并将解与真实解做了误差比较,
我们得到如下的结果:
\begin{table}[H]
	\centering
	\begin{tabular}{ccccc}
		\hline 
		网格密度$N$ & L2误差 & 阶数 & CPU时间(s) & CPU时间的阶数 \\ 
		\hline 
		10  & 6.00e-3 & / & 0.0018 & /  \\
		\hline 
		20 & 1.40e-3 & 2.09 & 0.3278 & 7.5 \\
		\hline 
		40 & 3.31e-4 & 2.08 & 13.7198 & 5.39 \\
		\hline 
		80 & 8.06e-5 & 2.04 & 319.70 & 4.54\\ 
		\hline
	\end{tabular}
\end{table}
此处我们采用的CFL条件数为0.1,计算终止时间为1,进程数为4.
\par
虽然受到诺依曼边界条件的影响,我们的理论数值精度只有1阶,但实际计算的
结果显示由2阶精度。
\par
由于我们固定CFL条件数为0.1,因此实际计算量为
$O(\frac{N^3}{\tau})=O(N^5)$,与实际计算结果较为匹配。
\subsection{并行效率分析}
我们的测试函数仍然为\eqref{eq:testfunction},对不同的进程数,得到的计算
结果如下:
\begin{table}[H]
	\centering
	\begin{tabular}{ccccccccc}
	\hline
       进程数  & 1 & 2 & 3 & 4 & 5 & 6 & 7 & 8 \\
	\hline
	  CPU时间(s)&  50.8490 & 26.5723 & 17.8507 & 13.5908 & 14.0187 &
	  11.8521 & 10.2867 & 9.5289\\
	\hline 
	 加速比 & / & 1.91 &  2.85 & 3.74 & 3.63 & 4.29 & 4.94 &  5.34 \\
	\hline 
	 效率 & / & 95.7\%  & 95.0\% &  93.5\% & 72.5\% &71.5\% & 70.6\% & 
	 66.7\% \\
	 \hline 
 \end{tabular}
\end{table}
此处我们采用的CFL条件数为0.1,终止时间为1,网格密度$N=40$.
\par
通过这些数据,我们可以发现该程序的并行效率较高,但当核数多时效率会有所
下降。经过比较,该程序最合适的进程数应设置为4.
\section{上机报告总结}
此次上机作业,我们使用了MPI进行了第一次编程实践,对MPI的机理有了初步的
认识,并对网格进行了一定规则的划分,使其负载尽量均衡。
\par
此外,我们计算了精确的每个进程需要索取的数据,这是对并行传输数据的一个
简单的实践。
\end{document}

