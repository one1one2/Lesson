\documentclass[a4paper,  11pt]{ctexart}
\usepackage{srcltx,graphicx}
\usepackage{amsmath, amssymb, amsthm}
\usepackage{color}
\usepackage{lscape}
\usepackage{multirow}
\usepackage{psfrag}
\usepackage{diagbox}
\usepackage[hang]{subfigure}
\usepackage{float}
\usepackage[colorlinks,linkcolor=black,anchorcolor=blue,citecolor=green]{hyperref}

\newtheorem{theorem}{Theorem}
\newtheorem{lemma}{Lemma}
\newtheorem{definition}{Definition}
\newtheorem{comment}{Comment}
\newtheorem{conjecture}{Conjecture}

\newcommand\bbR{\mathbb{R}}
\newcommand\bbN{\mathbb{N}}
\newcommand\bbC{\mathbb{C}}
\newcommand\bx{\boldsymbol{x}}
\newcommand\dd{\,\mathrm{d}}

\newcommand\diag{\mathrm{diag}}
\newcommand\tr{\mthrm{tr}}

\setlength{\oddsidemargin}{0cm}
\setlength{\evensidemargin}{0cm}
\setlength{\textwidth}{150mm}
\setlength{\textheight}{230mm}

\newcommand\note[2]{{{\bf #1}\color{red} [ {\it #2} ]}}
%\newcommand\note[2]{{ #1 }} % using this line in the formal version

\newcommand\pd[2]{\dfrac{\partial {#1}}{\partial {#2}}}
\newcommand\od[2]{\dfrac{\dd {#1}}{\dd {#2}}}
\newcommand{\bm}[1]{\mbox{\boldmath{$#1$}}}

\begin{document}
\title{ {\Huge{并行求解三维偏微分方程}}
\\ 并行计算第三次上机作业}
\author{郑灵超}
\maketitle
\tableofcontents
\newpage
\section{问题介绍}
本次要求解的方程是一个三维的偏微分方程,其定义域为立方体
\[
   -\pi \leq x,y,z \leq \pi,
\]
方程形式为
\begin{equation}
	\label{eq:equation}
	-\Delta u + u^3 = f ,
\end{equation}
所给的边界条件为周期边界。
\section{算法介绍}
直接求解这个方程较为困难,我们此处采用迭代法求解,迭代格式如下:
\begin{equation}
	\label{eq:iteration}
	-\Delta u^{(n+1)} + \lambda u^{(n+1)} = 
	f - (u^{(n)})^3 + \lambda u^{(n)}, 
	\quad \lambda>0.
\end{equation}
迭代的初始值可以任取,比如我们取
\begin{equation}
	\label{eq:u0}
	u^{(0)}(x,y,z)=u_0(x,y,z)=0.
\end{equation}
而对于迭代格式对应的偏微分方程\eqref{eq:iteration}, 我们可以采用
Fourier变换的方法来求解。

对方程\eqref{eq:iteration}左右两边同时作用Fourier变换,
\begin{equation}
	\label{eq:Fourier}
	\widehat{-\Delta u^{(n+1)}} + \lambda \widehat{ u^{(n+1)} }
	=	\widehat{f} - \widehat{(u^{(n)})^3} + \lambda
	\widehat{u^{(n)}}.
\end{equation}
在实际计算中,我们将求解区域划分为$N\times N\times N$的正方体网格,
对函数$u$的Fourier变换即为对$u_{ijk}$的三维离散Fourier变换。

注意到
\begin{equation}
	-\widehat{\Delta u}_{ijk} =
	(i^2+j^2+k^2)\widehat{u}_{ijk}, 
\end{equation}
从而方程\eqref{eq:Fourier}可以写为
\begin{equation}
   (i^2+j^2+k^2+\lambda)\widehat{u}_{ijk}=
		\widehat{f} - \widehat{(u^{(n)})^3} + \lambda
	\widehat{u^{(n)}},
\end{equation}
从而我们的迭代算法如下:
\begin{enumerate}
	\item 计算迭代初值$u^{(0)}=u_0$,并令$n=0$。
	\item 计算迭代格式\eqref{eq:iteration}的右端项$f-(u^{(n)})^3 + 
		\lambda u^{(n)}$, 并做Fourier变换得到
	$	\widehat{f} - \widehat{(u^{(n)})^3} + \lambda
	\widehat{u^{(n)}}$。
\item 计算$\widehat{u^{(n+1)}}$,
	$$
	\widehat{u^{(n+1)}}_{ijk}=\frac{1}{i^2+j^2+k^2+\lambda} 
	(\widehat{f} - \widehat{(u^{(n)})^3} + \lambda
	\widehat{u^{(n)}}).
	$$
\item 利用Fourier逆变换计算$u^{(n+1)}$。
\item $n=n+1$。如果满足终止条件,则结束计算;否则回到第二步。
\end{enumerate}
\section{程序分析}
\subsection{程序实现}
我们用MPI结合FFTW中一维的DFT函数来实现三维的Fourier变换。
\subsubsection{进程的拓扑结构}
将求解区域划分为$N\times N\times N$的正方体网格,假设使用的进程数
$\text{size}=Np^3$,则每个进程的数据为
$M^3$,其中$M=\frac{N}{Np}$。 

我们按照 $z$从小到大,$y$从小到大,$x$从小到大的顺序来给进程编号和对进
程内的数据进行编号。
则$\text{rank}\%Np$表示进程在$x$方向的位置编号,
$(\text{rank}/Np)\%Np$表示进程在$y$方向的位置编号,
$\text{rank}/Np/Np$表示进程在$z$方向的位置编号。

从而,我们可以为每个方向建立一个通讯器。每个通讯器包括所有在这个方向处
于同一条直线的进程,将一共$Np^3$个进程划分为$Np^2$个组,每组包含$Np$个
进程。
\subsubsection{三维Fourier变换的实现}
由于三维的离散Fourier变换等价于三个方向的一维离散Fourier变换的复合,因
此我们可以通过对三个方向分别进行一维Fourier变换来实现三维的离散Fourier
变换。

以$x$方向DFT为例,$0,1,\dots,Np-1$这$Np$个进程位于同一条直线上,这些进程
一共包含了$M\times M\times N$的数据,共需要进行$M^2$次长度为$N$的一维
DFT。为了充分利用所有进程,我们把数据重新分配给每个进程,
让每个进程执行$\frac{M^2}{Np}$次DFT,再将数据重新传递给原来的进程,从
而完成了$x$方向的DFT。

由于$x$方向连续的数据在内存上也连续,但$y,z$方向不然
,从而接下来执行另外两个方向的DFT时,需要将数据重新排列。
此外,我们采用MPI\_Alltoall函数进行同一个通讯器的所有进程间的数据传递
,由于数据排列规则的问题,再传递结束之后需要进行一次数据的重新排列才
能调用FFTW的函数进行运行。这些重新排列的细节在此略去。
\subsection{程序评价}
我们这次采用的程序的优点有:
\begin{itemize}
	\item 运行时所有进程都执行FFT,而且每个进程的任务一致,负载较为均
		衡。
	\item 只使用了存数据的内存和一个临时变量的内存,没有浪费任何内存空
		间。
	\item 传递信息时只传递了需要的信息,没有浪费。
\end{itemize}
但还有以下几点不足:
\begin{itemize}
	\item 尝试使用MPI\_DataType来定义列向量,从而实现快捷的信息传递和
		接收,
		但由于MPI\_Alltoall函数需要传递给$i$号进程的信息严格位于传递给
		$j$号进程的信息之前$(i<j)$,从而这个方案无法生效
		(或许是我程序写的不对)。从而在我们的程序
		中只能采用先对数据进行重新排列,再进行传递,浪费了不少的执行效
		率。
\end{itemize}
\section{数值结果}
\subsection{误差和收敛性分析}
我们此次试验计算了几个有真解的数值算例,
\begin{equation}
	\label{eq:example1}
    u(x,y,z) = \sin(x) 
\end{equation}
得到的计算结果如下:
\begin{table}[H]
	\centering
	\begin{tabular}{ccc}
		\hline 
		网格密度$N$ & 误差 & CPU时间(s) \\
		16 & 4.64e-15 & 0.018 \\
		32 & 4.64e-15 & 0.100 \\
		64 & 4.61e-15 & 1.391 \\
		128 & 4.62e-15 & 20.22 \\ 
		\hline
	\end{tabular}
\end{table}
此处我们采用的进程数为8,$\lambda=10$,迭代步数为100步,迭代初始值
$u_0=0$。

对于另外一个算例,
\begin{equation}
	u(x,y,z)=\sin(x)+10
\end{equation}
经过实验发现,$\lambda=10$时格式不收敛,此处我们修改$\lambda = 1000$,
以确保其收敛性。仍然采用8进程,迭代100步,得到的结果如下:
\begin{table}[H]
	\centering
	\begin{tabular}{ccc}
		\hline
		网格密度$N$ & 误差 & CPU时间(s) \\
		16 & 1.25e-11 & 0.021 \\
		32 & 1.25e-11 & 0.077 \\
		64 & 1.25e-11 & 1.366 \\
		128 & 1.25e-11 & 20.22 \\
		\hline 
	\end{tabular}
\end{table}
可以发现只要参数$\lambda$取得足够大,就能保证格式的收敛性。而采用
之前提议的格式
\begin{equation}
	\label{eq:iteration2}
	-\Delta u^{(n+1)} = f - (u^{(n)})^3
\end{equation}
对测试函数$u(x,y,z)=\sin(x)$,在我个人试验中,是无法保证收敛性的。

但另一方面,参数$\lambda$取得过大会影响迭代的收敛速度,因此如何给出一
个合适的$\lambda$是一个问题。我个人给了一种方案,
选取右端项$f$的最大值的一个倍数,即
$$\lambda =Const * ||f||_{\infty}. $$
\subsection{并行效率分析}
为了分析并行计算的效率,我们使用了与之前不同的一台机器,以测试在
1,8,27进程下计算所需要的时间。我们采取的测试函数的真解为
\begin{equation}
	u(x,y,z)=1+\sin(x+y+z),
\end{equation}
网格剖分密度$N=108$,计算迭代步数为100,参数$\lambda=10$。在这些条件下
,得到的计算结果如下:
\begin{table}[H]
	\centering
	\begin{tabular}{cccc}
		\hline
		进程数目 & 计算时间(s) & 加速比 & 效率  \\
		1 & 41.92& / & /\\
		8 & 6.24& 6.72& 83.97\% \\
		27 & 4.55 & 8.21 & 34.12\% \\
		\hline
	\end{tabular}
\end{table}
由于在初始化的时候我们申请了所有的fftw\_plan,以便重复使用,所以可以预
期当迭代步数增加时,运行时间将得到一定优化,以下我们对之前相同的实验数
据,将迭代步数修改为1000,进行重新计算,结果如下:
\begin{table}[H]
	\centering
	\begin{tabular}{cccc}
		\hline
		进程数目 & 计算时间(s) & 加速比 & 效率  \\
		1 & 411.98& / & /\\
		8 & 62.64& 6.62& 82.74\% \\
		27 & 36.60 & 11.26 & 41.69\% \\
		\hline
	\end{tabular}
\end{table}
可以发现对27进程,确实随着迭代步数的增加,效率会提高。
\section{上机报告总结}
此次上机作业,我们针对一个偏微分方程,设计了一个迭代法求求解,再利用迭
代格式的性质,采用Fourier变换来求解这个迭代格式。
\par
本次的并行算法,我们采用了并行实现一维的FFT,再利用迭一维的FFT来完成三
维的FFT运算。
\par
FFT由于需要传递的信息量较大,本身是不适合作为并行算法的,本次实验结果
也显示其并行效率不是甚高。
\par
最后,由于原格式在测试下的不稳定性,我们对其进行了修改,最终所采用的格
式为\eqref{eq:iteration}。经实验表明,该格式不用去求解关于常数$C$的三
次方程,而且是一个收敛的数值格式,能得到比较好的结果。
\end{document}

