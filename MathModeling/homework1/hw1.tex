\documentclass[a4paper,  11pt]{ctexart}
\usepackage{srcltx,graphicx}
\usepackage{amsmath, amssymb, amsthm}
\usepackage{color}
\usepackage{lscape}
\usepackage{multirow}
\usepackage{psfrag}
\usepackage{diagbox}
\usepackage[hang]{subfigure}
\usepackage{float}
\usepackage[colorlinks,linkcolor=black,anchorcolor=blue,citecolor=green]{hyperref}

\newtheorem{theorem}{Theorem}
\newtheorem{lemma}{Lemma}
\newtheorem{definition}{Definition}
\newtheorem{comment}{Comment}
\newtheorem{conjecture}{Conjecture}

\newcommand\bbR{\mathbb{R}}
\newcommand\bbN{\mathbb{N}}
\newcommand\bbC{\mathbb{C}}
\newcommand\bx{\boldsymbol{x}}
\newcommand\dd{\,\mathrm{d}}

\newcommand\diag{\mathrm{diag}}
\newcommand\tr{\mthrm{tr}}

\setlength{\oddsidemargin}{0cm}
\setlength{\evensidemargin}{0cm}
\setlength{\textwidth}{150mm}
\setlength{\textheight}{230mm}
\renewcommand\thesection{\arabic{section}} 
\newcommand\note[2]{{{\bf #1}\color{red} [ {\it #2} ]}}
%\newcommand\note[2]{{ #1 }} % using this line in the formal version

\newcommand\pd[2]{\dfrac{\partial {#1}}{\partial {#2}}}
\newcommand\od[2]{\dfrac{\dd {#1}}{\dd {#2}}}
\newcommand{\bm}[1]{\mbox{\boldmath{$#1$}}}

\begin{document}
\title{数学模型HW1}
\maketitle
% \tableofcontents
% \newpage
\section{第一题}
\subsection{}
Conservation law:
\[
     u_t + \left(\frac{u^2}{2}-\mu u_x\right)_x=0,\quad x\in\bbR,\quad
     t>0.
\]

Flux function: 
\(
    \frac{u^2}{2}-\mu u_x.
\)
\subsection{}
Hopf-Cole变换:
\[ 
u = -a\pd{}{x}\ln \phi = -a\frac{\phi_x}{\phi}.
\]
将Hopf-Cole 变换带入到原方程,并利用$a=2\mu$,
\[
-2\mu\frac{\phi\phi_{xt}-\phi_x\phi_t}{\phi^2}
+4\mu^2\frac{\phi\phi_x\phi_{xx}-\phi_x^3}{\phi^3}
=-2\mu^2\frac{\phi^2(\phi_x\phi_{xx}+\phi\phi_{xxx}-2\phi_x\phi_{xx})
-2\phi\phi_x(\phi\phi_{xx}-\phi_x^2)
}{\phi^4},
\]
化简可得
\[
-2\mu\frac{\phi\phi_{xt}-\phi_x\phi_t}{\phi^2}
=-2\mu^2\frac{\phi\phi_{xxx}-\phi_x\phi_{xx}
}{\phi^2},
\]
即
\[ 
  \left(
  \frac{\phi_t}{\phi}-\mu\frac{\phi_{xx}}{\phi}
  \right)_x=0.
\]
因此
\[ 
   \frac{1}{\phi}\phi_t-\mu\frac{1}{\phi}\phi_{xx}=g(t)
\]
是一个只关于$t$的函数。


\section{第二题}
\subsection{}
\[ 
   \od{f}{t} = \pd{f}{t} + \pd{f}{q}\cdot\pd{q}{t}+\pd{f}{p}\cdot 
   \pd{p}{t}.
\]
根据守恒律,我们有
\[ 
\pd{f}{t} + \pd{(f\dot{q})}{q} + \pd{(f\dot{p})}{p} = 0.
\]
将上式展开,我们有
\[ 
\pd{f}{t} + \pd{f}{q}\cdot\pd{q}{t} + \pd{f}{p}\cdot\pd{p}{t} 
+ f\left( \pd{\dot{q}}{q}+\pd{\dot{p}}{p}\right)= 0.
\]
又由
\[
    \dot{p} = \pd{H}{q},\quad \dot{q} = -\pd{H}{p},
\]
因此
\[  
    \pd{\dot{q}}{q} + \pd{\dot{p}}{p} = -\pd{^2H}{p\partial
    q}+\pd{^2H}{q\partial p}=0.
\]
因此我们有
\[ 
   \od{f}{t} = \pd{f}{t} + \pd{f}{q}\cdot\pd{q}{t}+\pd{f}{p}\cdot 
   \pd{p}{t} = 0.
\]
\subsection{}
对于任意区域$\Omega$,
\[
\od{}{t} \int_\Omega f(t,q,p)\dd q\dd p = 
\int_\Omega \pd{f}{t} \dd q\dd p = \int_\Omega
-\nabla\cdot(f\dot{q},f\dot{p})
\dd q\dd p = -\int_{\partial\Omega} (f\dot{q},f\dot{p})\cdot
\overrightarrow{\bm{n}}\dd S
\]
当$\Omega$包含了$f$的支集时,根据上式我们可得

\[  
\od{}{t}\int f(t,q,p)\dd q\dd p  = 0.
\]

\section{第三题}
\subsection{}
\begin{align*}
  \begin{aligned}
\od{}{t}\int |u|^2\dd x &= \int \pd{}{t}|u|^2 \dd x \\
&= \int (u_t\bar{u}+u\bar{u}_t)\dd x \\
&= \int \left(\frac{\hat{H}}{i\hbar}u\bar{u}
-u\frac{\bar{\hat{H}}}{i\hbar}\bar{u}
\right)\dd x \\ 
&= 0
   \end{aligned}
\end{align*}
其中最后一步利用了$\hat{H}$是一个自伴算子。
\subsection{}
由上题我们可知
\[ 
   \int |u(x,t)|^2 \dd x = \int |u(x,0)|^2 \dd x, \quad \forall t\geq
   0.
\]
因此所求的$\varepsilon$满足
\[  
1 = \int_{\bbR^d} a^2e^{-2|x|^2/\varepsilon}\dd x =
a^2\left(\frac{\varepsilon\pi}{2}\right)^{d/2}.
\]
(此处利用了$\int_{\bbR^d} e^{-A|x|^2}\dd x = (\pi/A)^{d/2}$.)

因此 
\[   
\varepsilon = \frac{2}{\pi a^{4/d}}.
\]
\end{document}
